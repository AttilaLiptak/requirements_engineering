\section{Bevezetés}

Az IntelliSense 360 egy adaptív perifériás érzékelő- és beavatkozó rendszer, amely a jármű közvetlen környezetét folyamatosan figyeli, és a vezető támogatásával a közúti ütközések elkerülését szolgálja különféle közlekedési helyzetekben.

A koncepció általános célja a biztonság növelése vezetés közben; a tipikus esetsorok közé tartozik a sávváltás, a parkolás és a szűk helyeken történő manőverezés.

\subsection{Felhasználók}
A rendszert személygépkocsi vezetők használják. A célfelhasználók között mindennapi autóvezetők és tapasztalt sofőrök is szerepelnek, akik számára fontos a jármű környezetének pontos érzékelése, valamint a holttérben lévő járművek, gyalogosok és kerékpárosok időben történő észlelése.

A felhasználó továbbra is a jármű irányításáért felel, a rendszer csak figyelmeztetésekkel és kisebb beavatkozásokkal segíti őt.

\subsection{Működés}
Az IntelliSense 360 különböző, egymással kommunikáló modulokból áll:

\begin{itemize}
    \item \textbf{Adaptív holttérfigyelés:} radar- és kamerarendszer érzékeli a jármű holtterében közeledő objektumokat, és figyelmezteti a vezetőt.
    \item \textbf{Oldalirányú ütközés-előrejelzés és elkerülés:} ultrahangos szenzorokkal figyeli a jármű közvetlen környezetét, parkoláskor vagy szűk helyzetben enyhe fékezéssel beavatkozhat.
    \item \textbf{Perifériás figyelmeztetés gyalogosokra és kerékpárosokra:} városi környezetben a kamerák azonosítják a jármű útvonalába kerülő mozgó objektumokat, és irányhoz kötött figyelmeztetést adnak.
    \item \textbf{Kiszállási asszisztens:} parkoláskor érzékeli az ajtók közelében megjelenő mozgó járműveket vagy gyalogosokat, és ajtónyitáskor akusztikus vagy vizuális jelzést ad.
    \item \textbf{Sávtartás kereszteződési támogatással:} a járművet a sávban tartja, felismeri a kanyarodási szándékot, és elkerüli a téves vagy felesleges korrekciókat.
\end{itemize}

A rendszer adaptív: a figyelmeztetések és beavatkozások intenzitását a vezető reakcióidejéhez, vezetési stílusához és a környezeti feltételekhez igazítja (például rossz látási viszonyok esetén a radaradatokat részesíti előnyben).

\subsection{A rendszer működési határai}
Az IntelliSense 360 nem autonóm vezetési rendszer. Nem veszi át a jármű irányítását, és nem képes minden helyzetben elkerülni az ütközést csak támogatja a vezetőt a döntések meghozatalában. Nem avatkozik be nagy sebességnél, ahol a beavatkozás kockázatos lehet (például autópályán előzés közben).

A kiszállási asszisztens nem blokkolhatja az ajtónyitást, csak figyelmeztetést adhat.

A rendszer nem működik teljes értékűen, ha több szenzor hibás vagy a kommunikáció megszakad; ilyen esetekben csak jelzést ad a korlátozott működésről. A sofőr bármikor felülbírálhatja a rendszer beavatkozásait.

\subsection{Jelentőség}
A közúti balesetek jelentős részét a figyelmetlenségből, holttérben maradt járművekből, oldalirányú ütközésekből, valamint gyalogos- és kerékpárosforgalomból adódó veszélyhelyzetek okozzák.

Az IntelliSense 360 ezeket a kockázatokat csökkenti azáltal, hogy időben felismeri a potenciális veszélyeket, és célzott, a helyzethez illeszkedő figyelmeztetéseket ad, vagy szükség esetén enyhén beavatkozik. A rendszer alkalmazása hozzájárul a közlekedésbiztonság növeléséhez, az emberi hibák hatásának mérsékléséhez.

\subsection{Tervezett funkciók}

\subsubsection{Adaptív holttérfigyelés}
Feladata a jármű holttereinek figyelése szenzorok segítségével és a vezető figyelmeztetése, ha egy másik jármű azokat megközelíti. A jelzések illeszkednek a vezető reakcióidejéhez, amelyről korábban a rendszer adatokat gyűjt.

Ha sávváltáskor a szenzorok másik járművet érzékelnek a holttérben, akkor a rendszer hang- és vizuális jelzéssel, opcionálisan a kormány rezgésével figyelmezteti a vezetőt. Alacsony fényviszonyok esetén a radarszenzorokból nyert adatokat részesíti előnyben a rendszer, valamint a figyelmeztető jelzések hosszabban aktívak.

\paragraph{Kérdések:}
mi történik, ha többen használják az autót? mindenkinek reakcióidő profil? rövidebb időintervallum alapján gyűjtött adatok? az összes átlaga?

\subsubsection{Oldalirányú ütközés előrejelzés és elkerülés}
Ultrahangos érzékelők segítségével felismeri az oldalirányú ütközések lehetőségét a nem megfelelő sávváltások, vezetői figyelmetlenség és útviszonyok figyelembevételével.

Amennyiben szükséges, a rendszer közbeavatkozik az ütközések elkerüléséért. Ha a rendszer akadályt érzékel, akkor hang- vagy vizuális jelzést ad. Képes automatikus fékezésre, amennyiben egy másik jármű a biztonsági határ alatti távolságban közelíti meg a személygépkocsit.

A téves riasztások kiküszöbölésére bizonyos esetekben csak vizuális figyelmeztetést ad, de nem avatkozik közbe. Ha a vezető manuálisan korrigál, a rendszer megszűnteti a közbeavatkozást.

A rendszer egyszerre több akadályra is képes felhívni a figyelmet, de csak a legkritikusabbra reagál. Amennyiben időjárási viszonyok miatt csökken a szenzorból kinyert adatok megbízhatósága, a rendszer változtat a beavatkozás mértékén és erről figyelmeztetést is ad.

\subsubsection{Perifériás figyelmeztetés gyalogosokra, kerékpárokra}
Ez a funkció a kiszolgáltatottabb közlekedési szereplőket érzékeli kis sebességnél, elsősorban városi környezetben. Célja a vezető figyelmeztetése a gyalogosok, illetve kerékpárosok figyelmetlenségéből adódó lehetséges ütközésekre.

A rendszer a veszély irányában hang- és vizuális jelzést ad az objektum jellegének, mozgási irányának és sebességének figyelembevételével. Megfelelően kezeli, ha a vezető elmulasztja az irányjelzést és csak vizuális figyelmeztetést ad.

\subsubsection{Kiszállási asszisztens}
Feladata, hogy figyelmeztesse az utasokat, ha kiszálláskor az ajtónyitást egy másik objektum akadályozná vagy a kiszálló utast közelségével veszélyeztetné. Amennyiben veszélyt érzékel, ajtónyitásra hangjelzést ad. A kiszállást fizikailag nem akadályozhatja.

Csak az érintett ajtónál ad jelzést, minden ajtót külön kezel. A közeledő objektum típusát felismeri és ennek megfelelően növeli a jelzés intenzitását. Ha ajtónyitás közben jelentkezik veszély, akkor képes késleltetett figyelmeztetés adására. Éjszakai módban is képes működni, ilyenkor csökkentett erejű hangjelzést ad, inkább vizuális jelzésre támaszkodik.

\subsubsection{Sávtartás kereszteződési támogatással}
A sávtartás lehetővé teszi, hogy a vezetői figyelmetlenségből vagy kimerültségből adódó részleges sávváltások elkerülhetők legyenek. Ilyenkor a rendszer kormányzással közbeavatkozik és térkép-, illetve kameraadatok segítségével a sáv közepén tartja a járművet.

A kereszteződési támogatás kiegészíti ennek funkcionalitását a kereszteződések felismerésével, kanyarodás segítésével, ütközési veszélyre figyelmeztetéssel, illetve a jármű megfelelő sávba pozícionálásával a kereszteződések elhagyását követően.

A sávtartás érdekében a rendszer képes a kormányzás korrigálására. Figyelembe veszi az irányjelzést, ilyenkor engedi a sáv elhagyását. Amennyiben ütközési veszélyt érzékel, figyelmezteti a vezetőt, de nem avatkozik közbe.

Képes a sávok kereszteződését kezelni a kormányállás, valamint a jármű iránya alapján. Figyelembe veszi az ideiglenes útjelzéseket, útépítést. Amennyiben a vezető jelentősen közbeavatkozik, a korrekció megszűnik. Ha nincs felfestett sáv, a rendszer akkor is képes a járművet az út megfelelő részén tartani az útpadka, járdaszegély figyelembevételével.

A közvetlen kanyarodás utáni sávváltást felismeri, kezeli. Záróvonalat, stop táblát, egyéb közúti jelzést felismeri és annak megfelelően közbeavatkozik. Alacsony sebességnél csak vizuális támogatást ad.
